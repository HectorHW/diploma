\specialsection{ЗАКЛЮЧЕНИЕ}\label{sec:conclusion}

В процессе выполнения выпускной квалификационной работы были получены следующие результаты:

\begin{itemize}
	\item была реализована подсистема генерации данных, применяющая мутации и кроссинговер над байтовыми последовательностями и синтаксическими деревьями;
	
	\item были изучены подходы к трассировке программ, сделан выбор в пользу подхода, основанного на ptrace, который был в дальнейшем реализован;
	
	\item была реализована подсистема, осуществляющая выбор тестовых примеров и методов мутации;
	
	\item была реализована функциональность, позволяющая конфигурировать разработанную систему через единый файл;
	
	\item был реализован пользовательский интерфейс в виде терминальной псевдографики;
	
	\item был собран набор уязвимых программ, на которых было осуществлено тестирование системы.
\end{itemize}

Разработанная в рамках выпускной квалификационной работы система способна осуществлять фазз-тестирование программного обеспечения без необходимости модификации исследуемой программы. Разработанная система  может применяться для автоматического обнаружения уязвимостей и генерации для них тестовых примеров в исполняемых файлах, что может поспособствовать разработке более безопасных и надёжных решений. При дальнейшей доработке возможно внедрение разработанной системы в процессы поддержания информационной безопасности организаций, осуществляющих разработку программного обеспечения.

В результате выпускной квалификационной работы были выполнены
все поставленные задачи, цель работы была достигнута, а также получены
следующие компетенции (таблица 1).

\noindent Таблица 1 -- Таблица компетенций
\vspace{5pt}

{
\fontsize{12}{9}\selectfont
%header
\noindent \begin{tabular}{|P{2.2cm}|P{6.5cm}|P{6.5cm}|}
	\hline
{\bfseries\centering Шифр компетенции} &
	{\bfseries Расшифровка проверяемой компетенции} &
	{\bfseries Расшифровка освоения компетенции} \\
	\hline
\end{tabular}%content
\vspace*{-12pt}
\noindent \begin{longtable}[c]{|p{2.2cm}|p{6.5cm}|p{6.5cm}|}
\endfirsthead
\caption*{\raggedright\hspace{-10pt} Продолжение таблицы 1\vspace*{-35pt}}\\
\endhead
УК-1 &
Способен осуществлять поиск, критический анализ и синтез информации, применять системный подход для решения поставленных задач &
Изучение и выбор средств управления процессами в ОС Linux
\\ \hline

УК-2 &
Способен определять круг задач в рамках поставленной цели и выбирать оптимальные способы их решения, исходя из действующих правовых норм, имеющихся ресурсов и ограничений &
Проведение анализа и выбор подхода для подсистемы трассировки исследуемой программы
\\ \hline

УК-3 &
Способен осуществлять социальное взаимодействие и реализовывать свою роль в команде &
Во время работы над ВКР происходило постоянное взаимодействие с научным руководителем
\\ \hline

УК-4 & 
Способен осуществлять деловую коммуникацию в устной и письменной формах на государственном языке Российской федерации и иностранном(ых) языке(ах) &
Проведение устной и письменной коммуникации о выполнении поставленных в ВКР задачах
\\ \hline
УК-5 & 
Способен воспринимать межкультурное разнообразие общества в социально-историческом, этическом и философском контекстах &
Проведение устной и письменной коммуникации о выполнении поставленных в ВКР задачах
\\ \hline
УК-6 & 
Способен управлять своим временем, выстраивать и реализовывать траекторию саморазвития на основе принципов образования в течение всей жизни & 
Организация плана работы по реализации задач, необходимых для достижения цели ВКР
\\ \hline
УК-7 & 
Способен поддерживать должный уровень физической подготовленности для обеспечения полноценной социальной и профессиональной деятельности & 
Введение перерывов с переключением на физическую активность при работе над ВКР
\\ \hline
УК-8 & 
Способен создавать и поддерживать безопасные условия жизнедеятельности, в том числе при возникновении чрезвычайных ситуаций & 
Соблюдение норм и мер предосторожности при работе за компьютером
\\ \hline
ОПК-1 & 
Способен применять фундаментальные знания, полученные в области математических и (или) естественных наук, и использовать их в профессиональной деятельности & 
Проведение анализа поведения математических функций, применение методов принятия решений в условиях неопределённости
\\ \hline
ОПК-2 & 
Способен применять компьютерные/суперкомпьютерные методы, современное программное обеспечение, в том числе отечественного происхождения, для решения задач профессиональной деятельности & 
Разработка подсистем трассировки и мутации исследуемой программы
\\ \hline
ОПК-3 & 
Способен к разработке алгоритмических и программных решений в области системного и прикладного программирования, математических, информационных и имитационных моделей, созданию информационных ресурсов глобальных сетей, образовательного контента, прикладных баз данных, тестов и средств тестирования систем и средств на соответствие стандартам и исходным требованиям & 
Разработка подсистем трассировки и мутации исследуемой программы, а также интерфейса пользователя
\\ \hline
ОПК-4 & 
Способен участвовать в разработке технической документации программных продуктов и комплексов с использованием стандартов, норм и правил, а также в управлении проектами создания информационных систем на стадиях жизненного цикла & 
Разработка примеров и документирование кода приложения
\\ \hline
ОПК-5 & 
Способен инсталлировать и сопровождать программное обеспечение информационных систем и баз данных, в том числе отечественного происхождения, с учетом информационной безопасности & 
Установка и поддержка разработанной системы на устройстве пользователя
\\ \hline
ПК-1 & 
Преподавание по дополнительным общеобразовательным программам & 
Изучение дополнительных материалов согласно теме ВКР
\\ \hline
ПК-2 & 
Проверка работоспособности и рефакторинг кода программного обеспечения & 
Проверка программы на предмет ошибок и их исправление
\\ \hline
ПК-3 & 
Интеграция программных модулей и компонент и верификация выпусков программного продукта & 
Комбинация выделенных компонентов и абстракций в цельную систему
\\ \hline
ПК-4 & 
Разработка требований и проектирование программного обеспечения & 
Разделение системы на компоненты и их реализация
\\ \hline
ПК-5 & 
Оценка и выбор варианта архитектуры программного средства & 
Выделение интерфейсов и деление программы на компоненты
\\ \hline
ПК-6 & 
Разработка тестовых случаев, проведение тестирования и исследование результатов & 
Выработка набора тестовых примеров в виде уязвимых исполняемых файлов и тестирование системы на них
\\ \hline
ПК-7 & 
Обеспечение функционирования баз данных & 
Обеспечение функционирования баз данных
\\ \hline
ПК-8 & 
Оптимизация функционирования баз данных & 
Оптимизация функционирования баз данных
\\ \hline
ПК-9 & 
Обеспечение информационной безопасности на уровне базы данных & 
Обеспечение информационной безопасности на уровне базы данных
\\ \hline
ПК-10 & 
Выполнение работ по созданию (модификации) и сопровождению информационных систем, автоматизирующих задачи организационного управления и бизнес-процессы & 
Выполнение работ по созданию (модификации) и сопровождению информационных систем, автоматизирующих задачи организационного управления и бизнес-процессы
\\ \hline
ПК-11 & 
Создание и сопровождение требований и технических заданий на разработку и модернизацию систем и подсистем малого и среднего масштаба и сложности & 
Создание и сопровождение требований и технических заданий на разработку и модернизацию систем и подсистем малого и среднего масштаба и сложности
\\ \hline

\end{longtable}
}
