\specialsection{ЗАКЛЮЧЕНИЕ}\label{sec:conclusion}

В ходе производственной практики были выполнены все поставленные задачи: изученны подходы, применяемые в существующих фаззерах, изученны подходы, используемые для контроял процессов в ОС Linux. Были выполнены задачи на разработку, включающие в себя разработку подсистемы, осуществляющей мутации входных данных, спроектированна и разработана система, осуществляющая трассировку программы, подвергаемой тестированию, проведено тестирование системы на уязвимых образцах исполняемых файлов, а также частично подготовлен текст выпускной квалификационной работы. Исследования и разработка, производимые в рамках практики, выполнялись в соответствии с поставленным планом на регулярной основе. Выполнение работ в рамках производственной практики производилось в условиях регулярной коммуникации с научным руководителем.

В ходе производственной практики были выполнены все поставленные задачи, цель работы была достигнута, а также получены следующие компетенции (таблица 1).

\vspace{12pt}
\noindent Таблица 1 -- Таблица компетенций
\vspace{-5pt}
{
	\fontsize{13}{9}\selectfont
	\noindent \begin{longtable}[c]{|p{2.6cm}|p{6.2cm}|p{6cm}|}
		\hline
		Компетенция & Расшифровка компетенции & Описание приобретенных знаний, умений и навыков \\
		
		\endfirsthead
		\caption*{\raggedright Продолжение таблицы 1\vspace*{-35pt}}\\
		\hline
		Компетенция & Расшифровка компетенции & Описание приобретенных знаний, умений и навыков \\
		\endhead
		
		\hline
		УК-1 &
		Способен осуществлять поиск, критический анализ и синтез информации, применять системный подход для решения поставленных задач &
		Изучение и выбор средств управления процессами в ОС Linux, изучение подходов в существующих решениях
		\\ \hline
		
		УК-2 &
		Способен определять круг задач в рамках поставленной цели и выбирать оптимальные способы их решения, исходя из действующих правовых норм, имеющихся ресурсов и ограничений &
		Проведение анализа и выбор подхода для подсистемы трассировки исследуемой программы
		\\ \hline
		
		УК-6 & 
		Способен управлять своим временем, выстраивать и реализовывать траекторию саморазвития на основе принципов образования в течение всей жизни & 
		Организация плана работы по реализации задач, необходимых для достижения цели практики, составление детального графика работ и дальнейшее следование ему
		\\ \hline
		
		ОПК-4 & 
		Способен участвовать в разработке технической документации программных продуктов и комплексов с использованием стандартов, норм и правил, а также в управлении проектами создания информационных систем на стадиях жизненного цикла & 
		Разработка примеров и документирование кода приложения, разработка UML-диаграмм, оформление текста отчёта о практике
		\\ \hline
		
	\end{longtable}
}
