\specialsection{ВВЕДЕНИЕ}\label{sec:intro}

% актуальность - кибербез, убытки от атак, cloudbleed - надо же чем-то напугать людей

Кибербезопасность стала областью с постоянно растущими бюджетами с обеих сторон -- и с точки зрения убытков, понесённых компаниями от кибератак, и с точки зрения затрат на защиту и исследования в области информационной безопасности. Несмотря на большую роль человеческого фактора при проведении многих атак, классические методы, построенные на эксплуатации уязвимостей в программном обеспечении не теряют своей актуальности из-за возможности в случае обнаружения уязвимости в распространённой информационной системе проведения автоматизированных атак на большое число целей. Например, обнаруженная в 2017 году уязвимость cloudbleed, вызывавшая утечку данных из-за ошибки в html-парсере в сервисе Сloudflare, которым пользуются порядка 80\% сайтов сети Интернет \cite{cloudbleed}. Одним из подходов, позволяющих обнаружать подобного рода уязвимости является фаззинг.

% актуальность - языки программирования - системы сложные, с++ всё ещё в топе



% собственно, тема - фаззинг

Фаззинг -- подход к исследованию программы на наличие уязвимостей, заключающийся в автоматической генерации тестовых примеров и наблюдении за поведением программы на сформированных образцах данных с целью обнаружения ошибок работы с памятью, зависаний и другого интересного для исследователя поведения.

Цель настоящей работы - создать систему фаззинга программного обеспечения, использующую основные принципы генетических алгоритмов, которая не требует для своей работы модификации исследуемой программы.

Основные задачи, выполнение которых необходимо для достижения поставленной цели:

\begin{enumerate}
	\item изучить подходы к генерации данных в существующих фаззерах;
	
	\item изучить возможности по управлению процессами, предоставляемые операционной системой (далее -- ОС) Linux;
	
	\item разработать компонент системы, реализующий мутацию входных данных;
	
	\item разработать подсистему, осуществляющую трассировку выполняемой программы;
	
	\item создать систему экспорта результатов тестирования для возможности дальнейшего анализа внешними программами;
	
	\item протестировать систему на уязвимых образцах исполняемых файлов.
\end{enumerate}
