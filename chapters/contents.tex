\renewcommand{\cfttoctitlefont}{\hfill\bfseries}

\renewcommand{\contentsname}{\vspace*{-3em}\begin{center}
		\fontsize{14}{14}\MakeUppercase{оглавление}
	\end{center}}

\renewcommand{\cftsecleader}{\cftdotfill{1}} %  частота точек
\renewcommand{\cftsubsecleader}{\cftdotfill{1}} %  частота точек
\renewcommand{\cftsecfont}{\normalfont} % обычный шрифт (не жирный) для секций в оглавлении
\renewcommand{\cftsecpagefont}{\normalfont} % обычный шрифт для цифер

\renewcommand{\cftaftertoctitleskip}{-14pt} % должно влиять на расстояние после названия оглавления

\renewcommand{\cftbeforesecskip}{0pt} % расстояние между строчками оглавления. Добавляется к стандартному

\renewcommand\cftsecafterpnum{\vskip0pt}
\renewcommand\cftsubsecafterpnum{\vskip0pt}

\cftsetindents{subsection}{0pt}{24pt}
\cftsetindents{section}{0pt}{24pt}

\setcounter{tocdepth}{2} % включаем в toc только section и subsection

\tableofcontents
