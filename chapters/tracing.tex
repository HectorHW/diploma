\section{Трассировка}

Важным компонентом, значительно ускоряющим процесс фаззинга, является измерение покрытия кода программы при запуске очередного тестового примера. Существует несколько подходов для измерения покрытия:


\subsection{Статическая инструментация}

Статическая инструментация программы, полагающаяся на применение специальных библиотек и компиляторов (например, afl-gcc), добавляющих в программу инструкции, на которые затем ориентируется фаззер для точного выяснения траектории выполнения программы.  

Плюсом такого подхода является быстрота проведения фаззинга (например, в программе может быть искусственно выделена та или иная секция, подвергаемая тестированию в бесконечном цикле, за счёт чего отпадает необходимость в трате ресурсов на постоянный запуск новых процессов и загрузки библиотек).

Минус данного подхода состоит в необходимости наличия доступа к исходному коду программы и необходимости дополнительной работы с ним.

\subsection{Динамическая инструментация}

Динамическая инструментация программы полагается на использование методов, схожих с таковыми, применяемыми в отладчиках - для сбора информации о траектории выполнения программы применяются точки останова, в которых записывается состояние регистра счётчика команд. В отличие от предыдущего подхода, в данном случае возможна работа с уже готовым исполняемым файлом, мы можем и не иметь исходного кода исследуемой программы.

Проблемой динамической инструментации является серьёзное влияние на скорость выполнения программы. 

Для снижения этого влияния могут применяться различные методы, например Coverage-guided tracing.